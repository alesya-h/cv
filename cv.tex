%%%%%%%%%%%%%%%%%%%%%%%%%%%%%%%%%%%%%%%%%
% Medium Length Professional CV
% LaTeX Template
%
% Original author:
% Trey Hunner (http://www.treyhunner.com/)
%
% Important note:
% This template requires the resume.cls file to be in the same directory as the
% .tex file. The resume.cls file provides the resume style used for structuring the
% document.
%
%%%%%%%%%%%%%%%%%%%%%%%%%%%%%%%%%%%%%%%%%

%----------------------------------------------------------------------------------------
%   PACKAGES AND OTHER DOCUMENT CONFIGURATIONS
%----------------------------------------------------------------------------------------

\documentclass{resume} % Use the custom resume.cls style

\usepackage[left=0.75in,top=0.6in,right=0.75in,bottom=0.6in]{geometry} % Document margins

\name{Ales Huzik} % Your name
\address{Sydney, Australia} % Your address
\address{+61~427~990-909 \\ me@aguzik.net} % Your phone number and email

\begin{document}

%----------------------------------------------------------------------------------------
%   EDUCATION SECTION
%----------------------------------------------------------------------------------------

\begin{rSection}{Education}

{\bf Belarusian State University of Informatics and Radioelectronics, Minsk} \hfill {\em July 2013} \\ 
Faculty of Computer Systems and Networks \\
M.S. in Computer Science \& Software Engineering \smallskip \\
{\bf Belarusian State University of Informatics and Radioelectronics, Minsk} \hfill {\em July 2012} \\ 
Faculty of Computer Systems and Networks \\
B.S. in Systems Engineering \\

\end{rSection}

\begin{rSection}{Certification}

{\bf CO0401EN } Beyond the Basics: Istio and IBM Cloud Kubernetes Service \hfill {\em December 2019} \\

\end{rSection}

%----------------------------------------------------------------------------------------
%   TECHNOLOGY SECTION
%----------------------------------------------------------------------------------------

\begin{rSection}{Have experience with}

\begin{tabular}{ @{} >{\bfseries}l @{\hspace{6ex}} l }
Operating Systems     & Linux (NixOS, Archlinux, Debian, CentOS, Ubuntu), Mac OS X \smallskip \\
Programming Languages & Clojure, Ruby, C, Bash, JavaScript, Perl, Common Lisp, Erlang, \\
                      & Maude, Factor, Lua, Terra, C++, Python and some others \smallskip \\
Libraries and Frameworks & React, Reagent, core.async, clojure.spec, prismatic/schema, \\
                      & mount-lite, Compojure, Kioo, Enlive
                      % & Rails, Sinatra, Bootstrap, Material UI
                      \smallskip \\
Markup and Typesetting & \LaTeX, HTML, Haml, Slim, CSS, SASS/SCSS, Less, Bootstrap \smallskip \\
Cloud                 & AWS (EC2, S3, IAM, RDS, DynamoDB, Route 53, EKS, ECS), \\
                      & Docker, Kubernetes, DigitalOcean, Dokku, Heroku \smallskip \\
Server technologies   & NixOps, Consul, Chef, Nginx, Apache httpd, statsd \\
                      & Postfix, Prosody, vsftpd, Squid, Corosync, Pacemaker, DRBD \smallskip \\
SQL Databases & PostgreSQL, MySQL/MariaDB, SQLite \smallskip \\
  NoSQL Databases & Datomic, DynamoDB, Neo4j, OrientDB, MongoDB, Cassandra \smallskip \\
% Tools & zsh, git, emacs, vim, tmux, gpicker, ctags, the silver searcher, \\
%       & GNU Global, splint, gdb, strace, valgrind, gprof \smallskip \\
Programming paradigms & Imperative, Object-oriented (class-based, prototype-based), \\
                      & Functional, Concatenative (stack-based), Logical (predicate logic, \\
                      & rewriting logic).
\end{tabular}

\end{rSection}

%----------------------------------------------------------------------------------------
%   WORK EXPERIENCE SECTION
%----------------------------------------------------------------------------------------

\begin{rSection}{Work positions}

\begin{rSubsection}{Atlassian, Confluence Server}{January 2018 - Present}{Senior Developer (Synchrony, Clojure)}{Sydney, Australia}
\item Analyzed L1-L2 support tickets related to the project, identified that majority of them are related to a troublesome Synchrony server setup in the Data Center configuration, came up with the solution to implement automatic management of Synchrony by Confluence in DC to eliminate the need for separate Synchrony server setup, and collaborated with Confluence Server Scale team to implement it.
\item Fork Synchrony to maintain a version for Confluence server, separate from Confluence Cloud, as teams are very far and don't communicate, and projects have very different infrastructure and needs. Removed cloud-specific build steps, code and configurations (docker cloud tests, building PaaS jar, cloud loadtest, s3 and dynamodb storage backends, redis caching, cloud-specific encryption, ec2 elb node discovery, aws logging, PaaS statsd, cloud metrics, etc.)
\item Build and development setup simplifications and improvements (move to just leiningen instead of bash+gulp+leiningen, remove most leiningen profiles, don't compile java, don't preprocess JS)
% \item Migrate from .lein-env to just environment variables for configuration
\item Refactored the code from component to mount-lite. This led to significant simplification of the code, with more explicit graph of dependencies, and significantly improved ability to work with code interactively from the repl.
\item Collected all configuration decisions in a single namespace. Previously many configuration options existed in a configuration namespace, but a lot of things have been using environ directly or using other means to detect certain runtime configurations (e.g. running in a cloud PaaS environment, on dev machine, in loadtest etc.). This allowed to ensure that differences between code paths in dev, test and prod environments are explicit and kept to a very minimum.
\item Implemented generic managed cluster-shared state as atoms on top of hazelcast's IMap. This allowed working with this distributed state the same way as if it were a normal local clojure atom, getting notifications when the state changes, using all clojure standard library functions for atoms, etc.
\item Found a workaround for a bug in core.async that causes exceptions to be thrown from an incorrect stack frame (https://clojure.atlassian.net/browse/ASYNC-198)
\item Implemented hub locking for the data eviction project.
\item Created simple and extendable cli tooling for all development, build, test, release and other tasks.
\item Worked with support team to investigate customer issues
\item Introduced Renovate to automatically manage dependency version upgrades in Confluence Server
~
\end{rSubsection}

\begin{rSubsection}{Atlassian, Confluence Cloud}{December 2016 - January 2018}{Senior Developer (Synchrony, Clojure)}{Sydney, Australia}

\item Code health improvements (code reviews, eliminating tech debt, improving development workflow)
\item GDPR and data eviction project
\item Pushed for changing synchronization data from linear to tree format for the new editor integration, so standard operational transformation logic won't break the document structure. Implemented a tree diffing algorithm to support tree format synchronization
\item Proposed and added jvm memory consumption metrics, so we can see when GC happens and can understand it's implications on the dynamic behavior of the system
\item Decreased application bundle size twice (from 100mb to 50mb), which made deployments noticeably faster
\item Calculated cluster startup dynamodb usage and increased limits accordingly, which allowed deploying during peak hours with no downtime (previously, deploying during peak hours could lead to downtime up to half an hour).
\item Designed and implemented an automated versioning project
\item Participated in interviewing potential candidates
\item Have been helping with onboarding of new team members
\item Led synchrony architecture bootcamp
\item Participated in the team on-call rotation
\item Participated in a company-wide hackathon (Ship-It) and got into finals
~
\end{rSubsection}

\begin{rSubsection}{Filemporium/Ourmedian}{July 2015 - December 2016}{Lead Clojure Developer}{Remote via Upwork}
\item Interviewed potential candidates
\item Regularly did code reviews
\item Documented project structure, project-specific code conventions, technical decisions,
  troubleshooting, and Amazon S3 project-specific step-by-step configuration guide
\item Did pair programming (to assist others with complicated tasks, to share project knowledge, to get
  back on track when I'm stuck)
\item Revised architecture in a way that drastically simplified client-side
  state management and allowed live page update of all active user sessions
\item Reengineered project build system using boot (previously leiningen were used).
  Fixed project build time (full recompilation now takes just a couple of minutes
  instead of an hour). Adjusted project code to work with reloaded workflow.
\item Refactored most of the project, implemented lots of functionality and fixed lots of bugs (e.g added
  config schema validation, cleaned up garbage logging (like {\tt (println "!!! FOO:" x)}) and implemented
  propper configurable logging throughout the system, implemented chunked file upload with an automatic reconnection, etc.)
\item Setup temporary deplyoment via docker and dokku
\item Implemented production-ready multiserver setup with zero-downtime deployment using NixOS/NixOps and Consul
\item Let go of a programmer that have been writing terrible code
\end{rSubsection}

\begin{rSubsection}{Filemporium/Ourmedian}{May 2015 - July 2015}{Clojure Developer}{Remote via Upwork}
\item Automated design updates
\item Added compile-time template checks to kioo templating library to ensure
  component correctness after a design update
\item Added support for using arbitrary npm libraries from ClojureScript code (to be
  able to utilize existing js React components)
\item Started writing project documentation. Documented actions needed to setup
  a project, update the design, add an npm library

% ~
\end{rSubsection}

\begin{rSubsection}{Softswiss Casino Software}{October 2014 - May 2015}{Senior Software Engineer}{Minsk, Belarus}
\item Implemented integrations with external game providers (CasinoTechnology, Fengaming)
\item Implemented completely custom design for a new customer (HTML/CSS)
\item Worked on an external wallet api implementation
\end{rSubsection}

\begin{rSubsection}{Rubyroid Labs, LLC}{April 2014 - September 2014}{Senior Software Engineer/Team Leader}{Minsk, Belarus}
\item Designed application architecture
\item Managed project development
\item Did code reviews
\item Solely implemented some internal services
\end{rSubsection}

\begin{rSubsection}{Intetics Co.}{July 2013 - April 2014}{Senior Software Engineer}{Minsk, Belarus}
\item Made fully-automated production server setup
\item Worked on refactoring legacy codebase
\item Worked on security-related features (IP whitelisting, XSS testing)
\item Implemented backend service for mobile apps.
\item Implemented automatic management of VPN servers DNS rotation
\item Did code reviews
\end{rSubsection}

\begin{rSubsection}{\parbox[t][2em][t]{9cm}{Belarusian State University of Informatics and Radioelectronics}}{February 2013 - January 2014 }{Teaching assistant at Electronic Computing Machines Department (part-time)}{Minsk, Belarus}
\item Taught first-year students programming in C.
\item Taught fourth-year students IP networking.
\item Taught students how to use Git and GitHub.
\item Together with students formalized grading criteria.
\item Formalized some code quality metrics.
\item Regularly reviewed students' code.
\item Taught Linux for interested students in my spare time.
\end{rSubsection}

\begin{rSubsection}{PowerMeMobile, Inc.}{January 2013 - February 2013}{Problem solver}{Minsk, Belarus}
\item Gave an idea of automating deployment process (new tier deployments may take up to
  a month of SysAdmin team work).
\item Implemented initial stages of deployment automation (installing base cluster software,
  configuring corosync/pacemaker, installing and configuring DRBD and nginx as resource agents) using Chef.
\item Made entire deployment configurable from a single place (from chef workstation using node attributes).
\item Got an agreement on opensourcing this efforts.
\end{rSubsection}

\begin{rSubsection}{Altoros Systems, Inc.}{October 2011 - September 2012}{Software Engineer in Ruby department}{Minsk, Belarus}
\item Proved that custom multisite functionality is a bad idea. Dropped the hacks and refactored application to use rails 3 engines.
\item Participated in porting internal RightScale services (mostly sinatra+cassandra) to JRuby to utilize native Thrift.
\item Participated in all stages of design and development on many projects.
\end{rSubsection}

\begin{rSubsection}{Itransition, Inc.}{February 2011 - October 2011}{Junior Developer in Ruby department}{Minsk, Belarus}
\item Solely ported large social gaming engine from Rails 2 to Rails 3.
\item Initiated using SCSS and Compass, which led to stylesheets development and modification speedup.
\item Configured production server from scratch and setup automated Capistrano deployment.
\end{rSubsection}

\end{rSection}

%----------------------------------------------------------------------------------------
%   NOTES SECTION
%----------------------------------------------------------------------------------------

\begin{rSection}{Some facts to better understand what kind of person I am}
  \smallskip
  \begin{list}{$\cdot$}{\leftmargin=0em} % \cdot used for bullets, no indentation
    \itemsep -0.5em \vspace{-0.5em} % Compress items in list together for aesthetics
  \item I decided to tie my work to computers when I was 5.
  \item First program in BASIC at age of 11, first HTML and JavaScript at 12, first program in Pascal at 13.
  \item I started playing with Linux when I was 14 (it was Mandrake 10 in 2005)
  \item I use Dvorak keyboard layout
  \item I use NixOS, and maintain a few packages in it. Before NixOS I used to use Archlinux.
  \item For many years I used to use tiling window managers and a very minimalistic desktop setup. I switched to Gnome Shell just over a year ago though.
  % \item I tend to automate everything I could.
  \item I use Emacs since 2010. Since 2011 I use it with Evil (vim emulation layer)
  \item I have dozens of personal opensource projects and have contributed to
        upstream of at least 20 other.
  % \item Some time ago I was passionate about japanese animation, so I learned
  %       some japanese and passed an international exam (JLPT4 certificate). I remember myself
  %       learning Esperanto and Toki Pona, and now I am learning Lojban.
  % \item I am an active ACM and ACM SIGPLAN member.
  \item One of my primary interests is expressivity of programming languages. I hope to study it as a PhD student one day.
  % \item I write tests very rarely and when I feel I wish to write one, I know that
  %   I should search for more simple solution that will be obviously working and not
  %   require any iterative validation. I do not believe that writing tons
  %   of tests may lead to good system design. More often it leads to design that is hard to change
  %   in any meaningful way. Instead of going test-first I go think-first, and don't start writing
  %   code until I clearly know what I'm going to write and why.
  \end{list}
\end{rSection}

%----------------------------------------------------------------------------------------
%   INTERESTS SECTION
%----------------------------------------------------------------------------------------

\begin{rSection}{Growth directions}
  \smallskip
  \begin{list}{$\cdot$}{\leftmargin=0em} % \cdot used for bullets, no indentation
    \itemsep -0.5em \vspace{-0.5em} % Compress items in list together for aesthetics
  \item Statistics and Machine Learning
  \item Semantics of programming languages
  % \item Programming languages with a minimalistic syntax: Lisp (Clojure, Common Lisp, Scheme,
  %       Qi/Shen), FORTH, Factor, APL, Tcl, Refal, Rebol, Smalltalk
  % \item Rewriting logic (Maude system in particular)
  \item Programming music (Overtone, Extempore) and visuals (Quil, Processing, Fluxus)
  \item Creating electronic music
  % \item Learning to draw
  % \item Neuroscience
  \end{list}
\end{rSection}

%----------------------------------------------------------------------------------------

\end{document}
