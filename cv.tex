%%%%%%%%%%%%%%%%%%%%%%%%%%%%%%%%%%%%%%%%%
% Medium Length Professional CV
% LaTeX Template
%
% Original author:
% Trey Hunner (http://www.treyhunner.com/)
%
% Important note:
% This template requires the resume.cls file to be in the same directory as the
% .tex file. The resume.cls file provides the resume style used for structuring the
% document.
%
%%%%%%%%%%%%%%%%%%%%%%%%%%%%%%%%%%%%%%%%%

%----------------------------------------------------------------------------------------
%   PACKAGES AND OTHER DOCUMENT CONFIGURATIONS
%----------------------------------------------------------------------------------------

\documentclass{resume} % Use the custom resume.cls style

\usepackage[left=0.75in,top=0.6in,right=0.75in,bottom=0.6in]{geometry} % Document margins

\name{Alesya Huzik} % Your name
\address{Surry Hills, NSW 2010, Australia} % Your address
\address{+61~427~990-909 \\ mail@alesya.me} % Your phone number and email
% Your address
\begin{document}

%----------------------------------------------------------------------------------------
%   EDUCATION SECTION
%----------------------------------------------------------------------------------------

\begin{rSection}{Highlights}
{\bf 8.5 years} of working with {\bf Clojure} \hfill {\em since May 2015} \\
{\bf 14 years} of commercial software development experience \hfill {\em since February 2011} \\
Have experience working in technical leadership roles (including {\bf Software Architect}, {\bf Team Lead}, and {\bf Lead developer} roles). \\
{\bf Master of Science} degree in {\bf Computer Science \& Software Engineering} \hfill {\em July 2013} \\ 
{\bf Bachelor of Science} degree in {\bf Systems Engineering} \hfill {\em July 2012} \\

\end{rSection}

%----------------------------------------------------------------------------------------
%   TECHNOLOGY SECTION
%----------------------------------------------------------------------------------------

\begin{rSection}{Have experience with}

\begin{tabular}{ @{} >{\bfseries}l @{\hspace{6ex}} p{11cm} }
Programming Languages & Clojure (8.5 years), Ruby (5 years),
                        JavaScript (over a decade here and there), TypeScript, C, Golang, Nushell, POSIX shell (bash, zsh), Python, Perl, Lua, C++, Erlang, Common Lisp, Vala, Rakudo, Factor, and some others
                        \smallskip \\
Operating Systems     & Linux (NixOS, Archlinux, Debian, CentOS,                                Ubuntu), Mac OS X, Windows (3.11 -- 11), DOS                            \smallskip \\
Libraries and Frameworks & React, Reagent, core.async, malli,
                           clojure.spec (+v2), prismatic/schema, jdbc.next, lib-noir, compojure, component, mount-lite, mount, Compojure, Kioo, Enlive, Rails, Sinatra, Bootstrap, Material UI
                           \smallskip \\
Cloud   & AWS (Cloudformation, Elastic Beanstalk, EC2,
          S3, IAM, RDS, DynamoDB, Route 53, EKS, ECS), DigitalOcean, 
          Cloudflare, Vultr, Nextcloud
          \smallskip \\
Linux, DevOps, networking &
     Docker/Podman, Docker Compose, Kubernetes, Terraform, Consul, Chef, libvirtd (kvm, libvirt-lxc), dokku, Nginx, Apache httpd, statsd, Postfix, Prosody, vsftpd, Squid, Corosync, Pacemaker, DRBD \smallskip \\
Markup and Typesetting & \LaTeX, HTML, Haml, Slim, CSS, SASS/SCSS,
                         Less, Bootstrap
                         \smallskip \\
Nix-based technologies & NixOS, nix flakes, Colmena, NixOps
                         \smallskip \\
SQL Databases & PostgreSQL, MySQL/MariaDB, SQLite
                \smallskip \\
NoSQL Databases & Redis, Datomic, DynamoDB, Neo4j,
                  OrientDB, MongoDB, Cassandra \smallskip \\
% Tools & zsh, git, emacs, vim, tmux, gpicker, ctags, the silver searcher, \\
%       & GNU Global, splint, gdb, strace, valgrind, gprof \smallskip \\
Programming paradigms & Imperative, Object-oriented (class-based,
                        prototype-based), Functional, Concatenative (stack-based), Logical (predicate logic, rewriting logic). \smallskip \\
Generative AI   & OpenAI model APIs, Anthropic Claude API,
                  Local LLMs (Ollama)
                  \smallskip \\
\end{tabular}

\end{rSection}

%----------------------------------------------------------------------------------------
%   WORK EXPERIENCE SECTION
%----------------------------------------------------------------------------------------

\begin{rSection}{Work positions}

\begin{rSubsection}{Payreq}{3 June 2024 - Present}{Principal Software Engineer}{Sydney, Australia} % 4 months

\item {\it (not disclosed due presently working there)}
% \item Updated the main app from Java 8 to Java 21, including dependencies, migrating from Tomcat-based
%   Elastic Beanstalk deployment to pure Java Elastic Beanstalk, updating bitbucket pipelines, updating
%   Cloudformation templates and migrating from Apache to NGINX reverse proxy.
% \item Rebased our peppol network service on top of upstream version of phase4 and made it possible to
%   update easier next time (it was forked without preserving available history, and there were around 200 commits
%   to upstream since the fork and around 200 commits on our side since).
~
\end{rSubsection}

\begin{rSubsection}{CIM}{25 September 2023 - 15 January 2024}{Senior Software Engineer}{Sydney, Australia} % 3.5 months
\item Lead the effort to migrate our internal SDK to use AWS SDK for Java v2, so we can migrate from JDK 11 to JDK 21.
    Updated the SDK, tests, and a few core services to use it.
\item Wrote a parser for a new meter data format
\item Reworked parser resolution from "magic" `requiring-resolve`-based to explicit one, so various Clojure
    tools would see a proper namespace dependency tree and not a bunch of dead code namespaces.
~
\end{rSubsection}

\begin{rSubsection}{Audience Republic}{February 2023 - 31 July 2023}{Software Architect}{Sydney, Australia} % 5 months
\item Redesigned URL shortening for SMS by defining a linear space for them and using format-preserving encryption.
  This addressed the bottleneck in unique short URL generation for escalating SMS volumes and enabled concurrent URL
  allocation without collision issues by being able to maintain a pool of ~1 million preallocated URLs for efficient,
  fail-safe usage.
%\item Designed linear-space short urls for SMS. When the number of SMS messages we were sending started to grow, generation of
%    unique short urls (for tracking purposes) became a bottleneck. Originally what it'd do is it'd try to insert a random string
%    of a specified length into the database and if that fails due a collision it'd retry up to 10 times. We couldn't generate
%    them in bulk because a collision inside a batch would make the entire insert to fail. Moreover if sending a batch would fail
%    we would waste those urls. The solution I came up with was to use a linear space for short urls and separate allocation and
%    use of short urls, using a format-preserving encryption algorithm to make allocated short urls look random. The background task
%    would keep a large number (about 1 million) of preallocated urls available for use, and any task that needed short urls could
%    just take however many it needs via `select ... limit ? for update skip locked`, so tasks can take them concurrently, and if
%    a task fails it's transaction would rollback releasing those short urls back into the pool.
\item switched all production servers from manual long-term SSL certificates to Let's Encrypt wildcard certificates with automatic
    refresh and sync across servers 
\item Designed and built a campaign bot that allows to simulate user activity, with time travel, for an arbitrary number of users on
    campaigns made for demo and sales purposes. This allows the sales team to create campaigns that mimic what that company would
    have and populate it with reasonable-looking user activity, so potential customers can see stats, graphs with activity
    distributions (sharing via social media etc.), referral registration percentages spanning days and months etc.
\item Embedded a code coverage library (cloverage) into the backend server request-handling code (non-production environment only),
    enabling direct code coverage measurement from UI-driven integration tests (which was the majority in our test suite).
    This approach revealed precise backend code paths activated during live interactions, significantly enhancing the precision and
    relevance of our test coverage analysis.
\item While working on the automation project I created a DSL wrapper
    around our resource DSL that was significantly more readable.
    Once other engineers discovered it they asked for it to be moved into the core of the project as a primary interface for working
    with db entities, which we gladly did.
\item Automation: new frontiers - search-based triggers. Automation became
    a popular part of the platform for customers, but they wanted
    to have an ability to trigger certain actions once a user starts or stops to match certain criteria (the platform already had very
    involved functionality for building multistage filters and saving them as segments). I designed and implemented search-based triggers
    in a way that was performant and didn't affect the rest of the platform. This was crucial since tables used for filters are the
    largest (few terabytes) and, despite a lot of denormalisation and other optimisation work, some segments are so complex they still require 
    several minutes to run (since customers are allowed to make them arbitrarily complex)
~
\end{rSubsection}

\begin{rSubsection}{Audience Republic}{July 2021 - February 2023}{Team Lead}{Sydney, Australia} % 1 year 7 months (19 months)
\item Leading the team working on the Automation project from inception to production release and beyond
\item Sourcing, interviewing, hiring candidates for the automation team
\item Training and mentoring new hires
\item Building and iterating on processes for managing the team (I was the first team lead at the company,
    previously there were just developers under the CTO)
\item Introduced JIRA in the company (previously a mix of Trello and Notion pages were used, JIRA started as
    a tool for my team and then became the main and only project management tool for the company)
\item Work estimation, planning, tracking team progress
\item Monitoring team health with 1-on-1 meetings, negotiating team needs with CTO and CEO (introducing
    paid leave and public holidays for engineers working remotely as contractors)
\item CNAME project (allowing customers to use customer-titled subdomains for their campaigns)
\item Wave (company's long shot experiment) -- a mobile app for one-to-many
    communications of artists/promoters
    with their fans. I designed the backend with quick prototyping and future
    scaling in mind (messaging
    backend was based on the Matrix protocol and Synapse was used as a
    messaging backend) and setup all necessary infrastructure. My team (plus
    a contractor mobile developer) built it. The project was released
    on Google Play and Apple AppStore, but was quickly deprioritised due 
    lack of adoption.
\item Migrated the project from an abandoned postgres.async SQL driver to standard JDBC (and jdbc.next clojure library).
    Wrote tools to find all SQL queries that rely on positional parameter numbers (postgres.async uses \$1, \$2 to refer
    to parameters, so unlike with JDBC they can be reused in a query), rewrote queries, added a shim layer that allowed
    both SQL backends to coexist by transparently translating old-style queries into the new style as long as all parameters
    are used in the query and are in a sequential order.
~
\end{rSubsection}

\begin{rSubsection}{Audience Republic}{January 2021 - July 2021}{Senior Backend Developer}{Sydney, Australia} % 7 months
\item Designed Automation subsystem of the product. It's a custom visual
    programming language for promoters to describe
    workflows that trigger in response to certain fan actions. E.g. "when fan 
    purchases a ticket to this event we want to add a tag to that fan,
    send them an email, wait until event date, send another email with an
    event reminder" etc.
\item Participated in sourcing, interviewing, hiring candidates
\item Found a solution to a scaling issue related to generation of tracking
      links when sending emails 
\item Designed a solution to scale integrations synchronisation without
      collisions %(select for update skip locked?)
\item Introduced database migrations via migratus library. Previously
    db changes were just written as SQL text in release notes and were only made by the CTO
\item When I joined the project had 2 "app" repositories and one "shared
    library" repository. This was leading to very complicated and slow
    process when "library" code was touched both during development and
    release processes (especially given the team was really small). I merged
    all 3 repositories in a way that preserved all the history of all three
    of them, so we ended up with a single Clojure project having multiple
    main namespaces and ability to still have `git blame` for every piece of
    the source code. Iteration became much easier and we never looked back.
\item Formed a team around me that ended up working on the automation project
~
\end{rSubsection}

\begin{rSubsection}{Atlassian, Confluence Server}{January 2018 - December 2020}{Senior Developer (Synchrony, Clojure)}{Sydney, Australia}
\item Analyzed L1-L2 support tickets related to the project, identified that majority of them are related to a troublesome Synchrony server setup in the Data Center configuration, came up with the solution to implement automatic management of Synchrony by Confluence in DC to eliminate the need for separate Synchrony server setup, and collaborated with Confluence Server Scale team to implement it
\item Fork Synchrony to maintain a version for Confluence server, separate from Confluence Cloud, as teams are very far and don't communicate, and projects have very different infrastructure and needs. Removed cloud-specific build steps, code and configurations (docker cloud tests, building PaaS jar, cloud loadtest, s3 and dynamodb storage backends, redis caching, cloud-specific encryption, ec2 elb node discovery, aws logging, PaaS statsd, cloud metrics, etc.)
\item Build and development setup simplifications and improvements (move to just leiningen instead of bash+gulp+leiningen, remove most leiningen profiles, don't compile java, don't preprocess JS)
% \item Migrate from .lein-env to just environment variables for configuration
\item Refactored the code from component to mount-lite. This led to significant simplification of the code, with more explicit graph of dependencies, and significantly improved ability to work with code interactively from the REPL
\item Collected all configuration decisions in a single namespace. Previously many configuration options existed in a configuration namespace, but a lot of things have been using environ directly or using other means to detect certain runtime configurations (e.g. running in a cloud PaaS environment, on dev machine, in loadtest etc.). This allowed to ensure that differences between code paths in dev, test and prod environments are explicit and kept to a very minimum
\item Implemented generic managed cluster-shared state as atoms on top of hazelcast's IMap. This allowed working with this distributed state the same way as if it were a normal local clojure atom, getting notifications when the state changes, using all clojure standard library functions for atoms, etc.
\item Found a workaround for a bug in core.async that causes exceptions to be thrown from an incorrect stack frame (https://clojure.atlassian.net/browse/ASYNC-198)
\item Implemented hub locking for the data eviction project
\item Created simple and extendable cli tooling for all development, build, test, release and other tasks
\item Worked with support team to investigate customer issues
\item Introduced Renovate to automatically manage dependency version upgrades in Confluence Server
~
\end{rSubsection}

\begin{rSubsection}{Atlassian, Confluence Cloud}{December 2016 - January 2018}{Senior Developer (Synchrony, Clojure)}{Sydney, Australia}

\item Code health improvements (code reviews, eliminating tech debt, improving development workflow)
\item GDPR and data eviction project
\item Pushed for changing synchronization data from linear to tree format for the new editor integration, so standard operational transformation logic won't break the document structure. Implemented a tree diffing algorithm to support tree format synchronization
\item Proposed and added jvm memory consumption metrics, so we can see when GC happens and can understand it's implications on the dynamic behavior of the system
\item Decreased application bundle size twice (from 100mb to 50mb), which made deployments noticeably faster
\item Calculated cluster startup dynamodb usage and increased limits accordingly, which allowed deploying during peak hours with no downtime (previously, deploying during peak hours could lead to downtime up to half an hour)
\item Designed and implemented an automated versioning project
\item Participated in interviewing potential candidates
\item Have been helping with onboarding of new team members
\item Led synchrony architecture bootcamp
\item Participated in the team on-call rotation
\item Participated in a company-wide hackathon (Ship-It) and got into finals
~
\end{rSubsection}

\begin{rSubsection}{Filemporium/Ourmedian}{July 2015 - December 2016}{Lead Clojure Developer}{Remote via Upwork}
\item Interviewed potential candidates
\item Regularly did code reviews
\item Documented project structure, project-specific code conventions, technical decisions,
  troubleshooting, and Amazon S3 project-specific step-by-step configuration guide
\item Did pair programming (to assist others with complicated tasks, to share project knowledge, to get
  back on track when I'm stuck)
\item Revised architecture in a way that drastically simplified client-side
  state management and allowed live page update of all active user sessions
\item Reengineered project build system using boot (previously leiningen were used)
  Fixed project build time (full recompilation now takes just a couple of minutes
  instead of an hour). Adjusted project code to work with reloaded workflow
\item Refactored most of the project, implemented lots of functionality and fixed lots of bugs (e.g added
  config schema validation, cleaned up garbage logging (like {\tt (println "!!! FOO:" x)}) and implemented
  propper configurable logging throughout the system, implemented chunked file upload with an automatic reconnection, etc.)
\item Setup temporary deplyoment via docker and dokku
\item Implemented production-ready multiserver setup with zero-downtime deployment using NixOS/NixOps and Consul
\item Let go of a programmer that have been writing terrible code
\end{rSubsection}

\begin{rSubsection}{Filemporium/Ourmedian}{May 2015 - July 2015}{Clojure Developer}{Remote via Upwork}
\item Automated design updates
\item Added compile-time template checks to kioo templating library to ensure
  component correctness after a design update
\item Added support for using arbitrary npm libraries from ClojureScript code (to be
  able to utilize existing js React components)
\item Started writing project documentation. Documented actions needed to setup
  a project, update the design, add an npm library

% ~
\end{rSubsection}

\begin{rSubsection}{Softswiss Casino Software}{October 2014 - May 2015}{Senior Software Engineer}{Minsk, Belarus}
\item Implemented integrations with external game providers (CasinoTechnology, Fengaming)
\item Implemented completely custom design for a new customer (HTML/CSS)
\item Worked on an external wallet api implementation
\end{rSubsection}

\begin{rSubsection}{Rubyroid Labs, LLC}{April 2014 - September 2014}{Senior Software Engineer/Team Leader}{Minsk, Belarus}
\item Designed application architecture
\item Managed project development
\item Did code reviews
\item Solely implemented some internal services
\end{rSubsection}

\begin{rSubsection}{Intetics Co.}{July 2013 - April 2014}{Senior Software Engineer}{Minsk, Belarus}
\item Made fully-automated production server setup
\item Worked on refactoring legacy codebase
\item Worked on security-related features (IP whitelisting, XSS testing)
\item Implemented backend service for mobile apps.
\item Implemented automatic management of VPN servers DNS rotation
\item Did code reviews
\end{rSubsection}

\begin{rSubsection}{\parbox[t][2em][t]{9cm}{Belarusian State University of Informatics and Radioelectronics}}{February 2013 - January 2014 }{Teaching assistant at Electronic Computing Machines Department (part-time)}{Minsk, Belarus}
\item Taught first-year students programming in C
\item Taught fourth-year students IP networking
\item Taught students how to use Git and GitHub
\item Together with students formalized grading criteria
\item Formalized some code quality metrics
\item Regularly reviewed students' code
\item Taught Linux for interested students in my spare time
\end{rSubsection}

\begin{rSubsection}{PowerMeMobile, Inc.}{January 2013 - February 2013}{Problem solver}{Minsk, Belarus}
\item Gave an idea of automating deployment process (new tier deployments may take up to
  a month of SysAdmin team work)
\item Implemented initial stages of deployment automation (installing base cluster software,
  configuring corosync/pacemaker, installing and configuring DRBD and nginx as resource agents) using Chef
\item Made entire deployment configurable from a single place (from chef workstation using node attributes)
\item Got an agreement on opensourcing this efforts
\end{rSubsection}

\begin{rSubsection}{Altoros Systems, Inc.}{October 2011 - September 2012}{Software Engineer in Ruby department}{Minsk, Belarus}
\item Proved that custom multisite functionality is a bad idea. Dropped the hacks and refactored application to use rails 3 engines
\item Participated in porting internal RightScale services (mostly sinatra+cassandra) to JRuby to utilize native Thrift
\item Participated in all stages of design and development on many projects
\end{rSubsection}

\begin{rSubsection}{Itransition, Inc.}{February 2011 - October 2011}{Junior Developer in Ruby department}{Minsk, Belarus}
\item Solely ported large social gaming engine from Rails 2 to Rails 3
\item Initiated using SCSS and Compass, which led to stylesheets development and modification speedup
\item Configured production server from scratch and setup automated Capistrano deployment
\end{rSubsection}

\end{rSection}

%----------------------------------------------------------------------------------------
%   NOTES SECTION
%----------------------------------------------------------------------------------------

\begin{rSection}{Random facts about me}
  \smallskip
  \begin{list}{$\cdot$}{\leftmargin=0em} % \cdot used for bullets, no indentation
    \itemsep -0.5em \vspace{-0.5em} % Compress items in list together for aesthetics
  \item I decided to tie my work to computers when I was 5
  \item First program in BASIC at age of 11, first HTML and JavaScript at 12, first program in Pascal at 13
  \item I started playing with Linux when I was 14 (it was Mandrake 10 in 2005)
  % \item I use Dvorak keyboard layout
  % \item I use NixOS, and maintain a few packages in it. Before NixOS I used to use Archlinux
  % \item For many years I used to use tiling window managers and a very minimalistic desktop setup. Few years ago I learned GNOME internals and switched to it.
  % \item I tend to automate everything I could
  % \item I use Emacs since 2010. Since 2011 I use it with Evil (vim emulation layer)
  \item I have dozens of personal open source projects and have contributed to at least 20 other
  % \item Some time ago I was passionate about japanese animation, so I learned
  %       some japanese and passed an international exam (JLPT4 certificate). I remember myself
  %       learning Esperanto and Toki Pona, and now I am learning Lojban
  % \item I am an active ACM and ACM SIGPLAN member
%  \item One of my primary interests is expressivity of programming languages. I hope to study it as a PhD student one day
  % \item I write tests very rarely and when I feel I wish to write one, I know that
  %   I should search for more simple solution that will be obviously working and not
  %   require any iterative validation. I do not believe that writing tons
  %   of tests may lead to good system design. More often it leads to design that is hard to change
  %   in any meaningful way. Instead of going test-first I go think-first, and don't start writing
  %   code until I clearly know what I'm going to write and why
  \end{list}
\end{rSection}

%----------------------------------------------------------------------------------------
%   INTERESTS SECTION
%----------------------------------------------------------------------------------------

\begin{rSection}{Interests and growth directions}
  \smallskip
  \begin{list}{$\cdot$}{\leftmargin=0em} % \cdot used for bullets, no indentation
    \itemsep -0.5em \vspace{-0.5em} % Compress items in list together for aesthetics
  \item Machine Learning and LLMs (I have some personal projects utilising them)
  % \item Semantics of programming languages
  % \item Programming languages with a minimalistic syntax: Lisp (Clojure, Common Lisp, Scheme,
  %       Qi/Shen), FORTH, Factor, APL, Tcl, Refal, Rebol, Smalltalk
  % \item Rewriting logic (Maude system in particular)
  \item Programming music (Overtone, Extempore) and visuals (Quil, Processing, Fluxus)
  \item Electronic music production
  % \item Learning to draw
  % \item Neuroscience
  \end{list}
\end{rSection}

%----------------------------------------------------------------------------------------

\end{document}

