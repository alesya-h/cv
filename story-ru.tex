% -*- coding: utf-8 -*-
\documentclass[14pt,a4paper]{extarticle}

\usepackage[T1,T2A]{fontenc}
\usepackage[utf8x]{inputenc}
\usepackage[english,russian]{babel}

\usepackage{indentfirst}
\usepackage[labelsep=period,justification=centering]{caption}
\usepackage{listings}
\lstset{
  basicstyle=\ttfamily,
  frame=tlBR
}

\usepackage{graphicx}
%\usepackage{tikz}
%\usetikzlibrary{arrows}
%\usetikzlibrary{fit}
%\usetikzlibrary{calc}
%\usetikzlibrary{shapes.geometric}
%\usetikzlibrary{shapes.symbols}
%%\usetikzlibrary{shapes.misc}
%\usetikzlibrary{positioning}
%\usetikzlibrary{decorations.pathmorphing}

\usepackage[14pt]{extsizes}

% интервал межстрочный: ~1.2
\renewcommand\normalsize{\fontsize{14}{22pt}\selectfont}

\renewcommand\labelitemi{\textendash}
\renewcommand\labelenumii{\asbuk{enumii})}

\clubpenalty=10000%
\widowpenalty=10000%

\usepackage[inner=2.2cm,top=2cm,outer=1.7cm,bottom=2.7cm,nohead]{geometry}

% \makeatletter
% \@newctr{figure}[section]
% \renewcommand \thefigure {\@arabic\c@section.\@arabic\c@figure}
% 
% \makeatother

\begin{document}

\section*{О себе (написано в 2012 или 2013 году)}

Гузик Алеся Юрьевна. Родилась 1 сентября 1990 года в Минске. С
компьютером с 5 лет. Около третьего класса твёрдо решила связать свою
жизнь с компьютерами. С 4 класса училась в физико-математической
школе. В 6 классе ходила на курсы по бейсику. Летом между 7 и 8
учила JavaScript и HTML по журналам <<Информатика>>\footnote{Доступа
к компьютеру не было, сохранился код написанный карандашом в тетради}.
В 8-9 классах помогала всему классу с паскалем. Где-то между 8 и 9
классами впервые познакомилась с Линуксом (Mandrake 10 на 3 CD).
После 9 класса поступила в лицейский физ-мат класс. В 10 классе писала
что-то на Delphi (где-то даже сохранился калькулятор для
арифметических выражений со скобками, написанный когда я ещё даже не
знала что такое обратная польская запись). Хотела научиться писать игры
(в игры при этом практически не играла). Пыталась самостоятельно учить
Java и C++, но быстро стало скучно. В 2007 году закончила лицейский
класс и поступила в Белорусский Государственный Университет Информатики
и Радиоэлектроники на факультет Компьютерных Систем и Сетей.

На первом курсе был Си, быстро разобравшись с которым я полезла на
википедию и начала читать всё подряд о языках программирования. Так я впервые узнала о
Ruby и позже о функциональном программировании. На тот момент кроме
Ruby мне показались очень интересными APL, J, Erlang, Haskell и
семейство Lisp. Отметила для себя что когда-нибудь всё это буду знать.
На первом курсе начала писать на Ruby генератор для выполнения
рассчётных работ по минимизации булевых функций\footnote{Написала
строк 300, до конца так и не довёла.}.

На втором курсе был C++, ассемблер для intel 8086 и позже C\#. В конце
давали несколько лекций по паттернам ООП. Паттерны показались
костылями из копипасты для недостаточно гибких языков. На втором курсе
активно взялась за полный переход на линукс. Параллельно разбиралась с Qt4
и слушала Ruby NoName Podcast. Начала учить японский. Начала ходить на
лекции инкубатора бизнес-проектов. Начала ходить на конференции (на тот
момент в основном связанные с веб-разработкой и юзабилити).

На третьем курсе взялась за ФП. Учила Erlang и Haskell. На эрланге даже
код какой-то сохранился. Сдала в Москве экзамен по японскому (JLPT4),
получила сертификат. Ходила на курсы по разработке под Android. Не
выдержала натиска бойлерплейтов на Java+XML и бросила. Где-то в это же
время начала использовать Emacs. Весной поставила на попробовать OS X.
Летом впервые съездила на конференцию Linux Vacation/Eastern Europe
(LVEE), куда с тех пор каждый год езжу. Сверстала свой первый курсовой
в \LaTeX.

На четвёртом курсе продолжила интересоваться ФП. Взялась смотреть
SICP, появился интерес к лиспам. На одной из конференций
пообщалась с рубистами и поняла что пора искать работу (ФП как вариант
работы не рассматривался, вакансий таких нет). Перечитала <<Programming
Ruby>>, начала читать <<Object-Oriented Javascript>>. С января пошла на курсы
в компанию itransition и после трёх занятий мне предложили работу.
На работе занималась разработкой на Ruby on Rails, аутсорсинг. К
этому времени я окончательно вернулась с OS X на Archlinux и начала
пользоваться тайловыми оконными менеджерами (XMonad, Awesome,
StumpWM и затем CLFSWM, которым пользуюсь до сих пор и являюсь
контрибьютеркой). Купила Nokia N900 и мучала её всем подряд\footnote{
Летом на сборах после военной кафедры когда лежала в медроте с
солдатами-срочниками в сосоедних палатах, прямо на ней написала руби-скрипт читавший данные
с акселерометра и начинавший громко орать mplayer'ом если его брали в
руки (чтобы телефон не украли)}. Учила из
интереса Perl. Написала на нём свой ack до того как узнала о его
существовании. Хотелось ФП и всё больше лиспа. Начала интересоваться
Clojure и читала про Scala. Пыталась продвинуть их использование в
компании. Идея с Clojure не понравилась совсем, на Scala
investigations дали часы чтобы отстала. Потом забыли.

В начале пятого курса начало маячить распределение и я решила что за те
деньги что мне платят ещё два года я работать не
хочу. Вспомнила что компания Altoros когда-то проводила конференции по
руби. Сходила к ним на собеседование и меня взяли на те деньги что
написала в качестве оптимальных, вышло в три раза больше. Купила VPS на
Linode. Подняла почту(postfix), jabber(prosody), ftp(vsftpd), http
proxy(squid), и веб-сервер(apache). Насильственно пересадила себя на
раскладку Дворака. Через месяц-полтора скорость набора достигла
прежнего уровня и продолжила расти. Перевела {\tt /home} со всеми
конфигами и проектами под git с подмодулями, после чего развернула копию своего
development-окружения прямо на сервере. Купила Thinkpad T520. Когда у
него сгорела материнская плата и он три недели был в ремонте,
весь девелопмент вёлся на сервере с ноутбука сестры через Putty. {\tt tmux}
проявил себя как нельзя лучше, снижение в производительности работы
было незначительным. Решила лучше разобраться с vim'ом. Поняла в чём
суть режимов и очень понравилось. Vim script по-прежнему казался
убогим. Нашёл evil для емакса, которым пользуюсь до сих пор. Vim с
минимальным конфигом использую для единичных файлов. Scala ушла на
второй план. Появился нездоровый интерес к Common Lisp и его
мультипарадигменности, макросам и ориентации на DSL. Читала <<ANSI Common
Lisp>> и <<On Lisp>> Пола Грэма. Поучаствовала в конкурсе <<Системный
администратор 2012>>, получила сертификат что являюсь <<гуру системного
администрирования>>. Веб-разработка на Ruby on Rails окончательно надоела своей
однообразностью. Решила уйти с работы. Поскольку распределена, решила
пойти в магистратуру и совместить приятное с полезным "--- в рамках
магистерской пишу свою реализацию Ruby на Common Lisp.

Сейчас заканчиваю курс Мартина Одерски <<Functional Programming in
Scala>> на {\tt coursera.com}, параллельно читаю <<Let over Lambda>> и
<<The Art of Metaobject Protocol>>. Начала разбираться с OCaml'ом. В
планах висит математическая логика (потом, возможно, теория типов и теория
категорий)

\section*{С чем работала по текущий момент}

\subsection*{Администрирование}

Работала с Debian, Ubuntu, CentOS, Fedora, SuSE,
Archlinux. Подготавливала сервера и код приложений для деплоймента через
Capistrano. Настраивала Postfix, VSFTPD, Apache/mod\_rails, NGINX,
Squid. Писала скрипты для всего что не хотелось делать
руками. Знаю IPv4-сети и немного IPv6. Настраивала загрузку в чистом
UEFI режиме. Настраивала сеть в UEFI shell. Держу две VPSки.

\subsection*{Системное ПО}

C89, C99, POSIX API(работа с процессами, файлами, IP- и UNIX-сокетами,
синхронизация процессов и потоков). Писала под x86/amd64, MSP430, ATMEL (Arduino).

\subsection*{Веб-разработка}

Писала на Ruby on Rails и Sinatra. Для генерации темплейтов использовала
HAML, Slim, Liquid. Для CSS использовала SASS/SCSS и Less. Использовала
CoffeeScript для генерации JS. Использовала RefineryCMS. Использовала
RSpec, Cucumber, Test::Unit, Capybara. Настраивала Passenger для
продакшена, настраивала деплоймент через Capistrano. Работала с MySQL,
PostgreSQL и MongoDB (с ней в минимальном объёме) в качестве базы
данных.

\subsection*{Функциональное программирование}

Понимаю что такое чистота, оптимизация хвостовой рекурсии, замыкания,
функции высших порядков, карринг, бесточечная нотация, модель акторов.
Писала на Erlang (помню gen\_server, supervisor; работала с yaws), OCaml (недавно),
Haskell (давно и совсем чуть-чуть), Scheme, Scala. Писала в функциональном стиле
на Ruby, JavaScript, Common Lisp. Есть желание расти в этом направлении.

\end{document}

