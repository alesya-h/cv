% -*- coding: utf-8 -*-
\documentclass[14pt,a4paper]{extarticle}

\usepackage[T1,T2A]{fontenc}
\usepackage[utf8x]{inputenc}
\usepackage[english,russian]{babel}

\usepackage{indentfirst}
\usepackage[labelsep=period,justification=centering]{caption}
\usepackage{listings}
\lstset{
  basicstyle=\ttfamily,
  frame=tlBR
}

\usepackage{graphicx}
%\usepackage{tikz}
%\usetikzlibrary{arrows}
%\usetikzlibrary{fit}
%\usetikzlibrary{calc}
%\usetikzlibrary{shapes.geometric}
%\usetikzlibrary{shapes.symbols}
%%\usetikzlibrary{shapes.misc}
%\usetikzlibrary{positioning}
%\usetikzlibrary{decorations.pathmorphing}

\usepackage[14pt]{extsizes}

% интервал межстрочный: ~1.2
\renewcommand\normalsize{\fontsize{14}{22pt}\selectfont}

\renewcommand\labelitemi{\textendash}
\renewcommand\labelenumii{\asbuk{enumii})}

\clubpenalty=10000%
\widowpenalty=10000%

\usepackage[inner=2.2cm,top=2cm,outer=1.7cm,bottom=2.7cm,nohead]{geometry}

% \makeatletter
% \@newctr{figure}[section]
% \renewcommand \thefigure {\@arabic\c@section.\@arabic\c@figure}
% 
% \makeatother

\begin{document}

\section*{О себе}

Гузик Алесь Юрьевич. Родился 1 сентября 1990 года в Минске. С
компьютером с 5 лет. Около третьего класса твёрдо решил связать свою
жизнь с компьютерами. С 4 класса учился в физико-математической
школе. В 6 классе ходил на курсы по бейсику. Летом между 7 и 8
учил JavaScript и HTML по журналам <<Информатика>>\footnote{Доступа
к компьютеру не было, сохранился код написанный карандашом в тетради}.
В 8-9 классах помогал всему классу с паскалем. Где-то между 8 и 9
классами впервые познакомился с Линуксом (Mandrake 10 на 3 CD).
После 9 класса поступил в лицейский физ-мат класс. В 10 классе писал
что-то на Delphi (где-то даже сохранился калькулятор для
арифметических выражений со скобками, написанный когда я ещё даже не
знал что такое обратная польская запись). Хотел научиться писать игры
(в игры при этом практически не играл). Пытался самостоятельно учить
Java и C++, но быстро стало скучно. В 2007 году закончил лицейский
класс и поступил в Белорусский Государственный Университет Информатики
и Радиоэлектроники на факультет Компьютерных Систем и Сетей.

На первом курсе был Си, быстро разобравшись с которым я полез на
википедию и начал читать всё подряд о языках программирования. Так я впервые узнал о
Ruby и позже о функциональном программировании. На тот момент кроме
Ruby мне показались очень интересными APL, J, Erlang, Haskell и
семейство Lisp. Отметил для себя что когда-нибудь всё это буду знать.
На первом курсе начал писать на Ruby генератор для выполнения
рассчётных работ по минимизации булевых функций\footnote{Написал
строк 300, до конца так и не довёл.}.

На втором курсе был C++, ассемблер для intel 8086 и позже C\#. В конце
давали несколько лекций по паттернам ООП. Паттерны показались
костылями из копипасты для недостаточно гибких языков. На втором курсе
активно взялся за полный переход на линукс. Параллельно разбирался с Qt4
и слушал Ruby NoName Podcast. Начал учить японский. Начал ходить на
лекции инкубатора бизнес-проектов. Начал ходить на конференции (на тот
момент в основном связанные с веб-разработкой и юзабилити).

На третьем курсе взялся за ФП. Учил Erlang и Haskell. На эрланге даже
код какой-то сохранился. Сдал в Москве экзамен по японскому (JLPT4),
получил сертификат. Ходил на курсы по разработке под Android. Не
выдержал натиска бойлерплейтов на Java+XML и бросил. Где-то в это же
время начал использовать Emacs. Весной поставил на попробовать OS X.
Летом впервые съездил на конференцию Linux Vacation/Eastern Europe
(LVEE), куда с тех пор каждый год езжу. Сверстал свой первый курсовой
в \LaTeX.

На четвёртом курсе продолжил интересоваться ФП. Взялся смотреть
SICP, появился интерес к лиспам. На одной из конференций
пообщался с рубистами и понял что пора искать работу (ФП как вариант
работы не рассматривался, вакансий таких нет). Перечитал <<Programming
Ruby>>, начал читать <<Object-Oriented Javascript>>. С января пошёл на курсы
в компанию itransition и после трёх занятий мне предложили работу.
На работе занимался разработкой на Ruby on Rails, аутсорсинг. К
этому времени я окончательно вернулся с OS X на Archlinux и начал
пользоваться тайловыми оконными менеджерами (XMonad, Awesome,
StumpWM и затем CLFSWM, которым пользуюсь до сих пор и являюсь
контрибьютером). Купил Nokia N900 и мучал её всем подряд\footnote{
Летом на сборах после военной кафедры когда лежал в медроте с
солдатами-срочниками прямо на ней написал руби-скрипт читавший данные
с акселерометра и начинавший громко орать mplayer'ом если его брали в
руки (чтобы телефон не украли)}. Учил из
интереса Perl. Написал на нём свой ack до того как узнал о его
существовании. Хотелось ФП и всё больше лиспа. Начал интересоваться
Clojure и читал про Scala. Пытался продвинуть их использование в
компании. Идея с Clojure не понравилась совсем, на Scala
investigations дали часы чтобы отстал. Потом забыли.

В начале пятого курса начало маячить распределение и я решил что за те
деньги что мне платят ещё два года я работать не
хочу. Вспомнил что компания Altoros когда-то проводила конференции по
руби. Сходил к ним на собеседование и меня взяли на те деньги что
написал в качестве оптимальных, вышло в три раза больше. Купил VPS на
Linode. Поднял почту(postfix), jabber(prosody), ftp(vsftpd), http
proxy(squid), и веб-сервер(apache). Насильственно пересадил себя на
раскладку Дворака. Через месяц-полтора скорость набора достигла
прежнего уровня и продолжила расти. Перевёл {\tt /home} со всеми
конфигами и проектами под git с подмодулями, после чего развернул копию своего
development-окружения прямо на сервере. Купил Thinkpad T520. Когда у
него сгорела материнская плата и он три недели был в ремонте,
весь девелопмент вёлся на сервере с ноутбука сестры через Putty. {\tt tmux}
проявил себя как нельзя лучше, снижение в производительности работы
было незначительным. Решил лучше разобраться с vim'ом. Понял в чём
суть режимов и очень понравилось. Vim script по-прежнему казался
убогим. Нашёл evil для емакса, которым пользуюсь до сих пор. Vim с
минимальным конфигом использую для единичных файлов. Scala ушла на
второй план. Появился нездоровый интерес к Common Lisp и его
мультипарадигменности, макросам и ориентации на DSL. Читал <<ANSI Common
Lisp>> и <<On Lisp>> Пола Грэма. Поучаствовал в конкурсе <<Системный
администратор 2012>>, получил сертификат что являюсь <<гуру системного
администрирования>>. Веб-разработка на Ruby on Rails окончательно надоела своей
однообразностью. Решил уйти с работы. Поскольку распределён, решил
пойти в магистратуру и совместить приятное с полезным "--- в рамках
магистерской пишу свою реализацию Ruby на Common Lisp.

Сейчас заканчиваю курс Мартина Одерски <<Functional Programming in
Scala>> на {\tt coursera.com}, параллельно читаю <<Let over Lambda>> и
<<The Art of Metaobject Protocol>>. Начал разбираться с OCaml'ом. В
планах висит математическая логика (потом, возможно, теория типов и теория
категорий)

\section*{С чем работал по текущий момент}

\subsection*{Администрирование}

Работал с Debian, Ubuntu, CentOS, Fedora, SuSE,
Archlinux. Подготавливал сервера и код приложений для деплоймента через
Capistrano. Настраивал Postfix, VSFTPD, Apache/mod\_rails, NGINX,
Squid. Писал скрипты для всего что не хотелось делать
руками. Знаю IPv4-сети и немного IPv6. Настраивал загрузку в чистом
UEFI режиме. Настраивал сеть в UEFI shell. Держу две VPSки.

\subsection*{Системное ПО}

C89, C99, POSIX API(работа с процессами, файлами, IP- и UNIX-сокетами,
синхронизация процессов и потоков). Писал под x86/amd64, MSP430, ATMEL (Arduino).

\subsection*{Веб-разработка}

Писал на Ruby on Rails и Sinatra. Для генерации темплейтов использовал
HAML, Slim, Liquid. Для CSS использовал SASS/SCSS и Less. Использовал
CoffeeScript для генерации JS. Использовал RefineryCMS. Использовал
RSpec, Cucumber, Test::Unit, Capybara. Настраивал Passenger для
продакшена, настраивал деплоймент через Capistrano. Работал с MySQL,
PostgreSQL и MongoDB (с ней в минимальном объёме) в качестве базы
данных.

\subsection*{Функциональное программирование}

Понимаю что такое чистота, оптимизация хвостовой рекурсии, замыкания,
функции высших порядков, карринг, бесточечная нотация, модель акторов.
Писал на Erlang (помню gen\_server, supervisor; работал с yaws), OCaml (недавно),
Haskell (давно и совсем чуть-чуть), Scheme, Scala. Писал в функциональном стиле
на Ruby, JavaScript, Common Lisp. Есть желание расти в этом направлении.

\end{document}

